\documentclass[12pt,a4paper]{article}
\usepackage[UTF8]{ctex}
\usepackage{amsmath}
\usepackage{graphicx}
\usepackage{float}
\usepackage{geometry}
\usepackage{booktabs}
\usepackage{hyperref}
\usepackage{listings}
\usepackage{xcolor}

\geometry{left=2.5cm,right=2.5cm,top=3cm,bottom=3cm}

\title{时间序列预测任务报告\\室外温度预测}
\author{数据挖掘课程作业}
\date{\today}

\begin{document}

\maketitle

\section{引言}

本报告针对气象数据的时间序列预测任务,使用深度学习模型预测室外温度(OT)的变化。数据集包含德国某气象站半年内的气象数据,共26200个数据点,时间间隔为每10分钟记录一次。本报告将从数据预处理、模型设计和模型评估三个方面详细介绍实验过程。

\section{数据预处理}

\subsection{数据集描述}

数据集包含20个气象特征指标,包括:
\begin{itemize}
    \item 气压相关:p (mbar), VPmax (mbar), VPact (mbar), VPdef (mbar)
    \item 温度相关:T (degC), Tpot (K), Tdew (degC), Tlog (degC)
    \item 湿度相关:rh (\%), sh (g/kg), H2OC (mmol/mol)
    \item 风速相关:wv (m/s), max. wv (m/s), wd (deg)
    \item 降水相关:rain (mm), raining (s)
    \item 辐射相关:SWDR (W/m$^2$), PAR (umol/m$^2$s), max. PAR (umol/m$^2$s)
    \item 其他:rho (g/m$^3$)
\end{itemize}

目标变量为室外温度(OT),数据总量为26200个样本。

\subsection{特征标准化}

由于不同特征具有不同的量纲和数值范围,直接使用原始数据会导致模型训练不稳定。因此,我们使用StandardScaler对特征进行标准化处理:

\begin{equation}
z = \frac{x - \mu}{\sigma}
\end{equation}

其中,$\mu$为特征的均值,$\sigma$为标准差。标准化后的特征均值为0,标准差为1,有助于模型收敛和提高预测精度。

\subsection{滑动窗口切分}

时间序列预测需要将历史数据转换为监督学习问题。我们采用滑动窗口方法,将时间序列数据转换为序列-标签对。

对于窗口大小为$w$的滑动窗口,我们使用过去$w$个时间点的特征来预测下一个时间点的目标值。具体而言:
\begin{itemize}
    \item 输入序列:$X_t = [x_{t-w+1}, x_{t-w+2}, \ldots, x_t]$,其中$x_i$为第$i$个时间点的20维特征向量
    \item 输出标签:$y_t = OT_{t+1}$,即下一个时间点的室外温度
\end{itemize}

在本实验中,我们使用窗口大小为24(对应4小时的历史数据,每10分钟一个时间点)来构建训练样本。通过滑动窗口切分,原始数据被转换为形状为$(N-w, w, 20)$的输入序列和形状为$(N-w,)$的目标标签,其中$N$为原始数据点数量。

\subsection{训练集与测试集划分}

考虑到时间序列数据的时序特性,我们采用时间顺序划分方法,确保训练集的时间早于测试集,避免数据泄露问题。具体划分方式如下:

\begin{itemize}
    \item 训练集:前80\%的数据,共20947个样本
    \item 测试集:后20\%的数据,共5237个样本
\end{itemize}

这种划分方式保证了模型在训练时只能看到历史数据,在测试时预测未来数据,更符合实际应用场景。

\section{模型设计}

\subsection{模型架构}

本实验设计了一个基于LSTM(Long Short-Term Memory)神经网络和注意力机制的时间序列预测模型,主要包含以下组件:


\subsubsection{LSTM层}

LSTM是一种特殊的循环神经网络(RNN),能够有效处理长序列依赖关系。本模型使用3层LSTM,隐藏层维度为256。LSTM的核心思想是通过门控机制(遗忘门、输入门、输出门)来控制信息的流动,从而解决传统RNN的梯度消失问题。

对于输入序列$X_t = [x_1, x_2, \ldots, x_w]$,LSTM层逐时间步处理,最终输出形状为$(batch\_size, w, 256)$的序列表示。

\subsubsection{注意力机制}

为了捕捉不同时间步对预测结果的重要性,我们在LSTM层后添加了注意力机制。注意力机制通过计算每个时间步的注意力权重,对LSTM输出进行加权求和,得到上下文向量:

\begin{equation}
\alpha_i = \frac{\exp(e_i)}{\sum_{j=1}^{w}\exp(e_j)}
\end{equation}

\begin{equation}
c = \sum_{i=1}^{w}\alpha_i \cdot h_i
\end{equation}

其中,$e_i$为第$i$个时间步的注意力分数,$h_i$为LSTM在第$i$个时间步的输出,$c$为最终的上下文向量。

\subsubsection{全连接层}

全连接层将注意力机制输出的上下文向量(256维)映射到最终的预测值。本模型使用三层全连接网络:
\begin{itemize}
    \item 第一层:256 $\rightarrow$ 128,ReLU激活函数,Dropout=0.3
    \item 第二层:128 $\rightarrow$ 64,ReLU激活函数,Dropout=0.3
    \item 第三层:64 $\rightarrow$ 1,线性输出
\end{itemize}

Dropout层用于防止过拟合,提高模型的泛化能力。

\subsection{模型参数}

模型的主要超参数设置如下:
\begin{itemize}
    \item 输入维度:20(特征数量)
    \item LSTM隐藏层维度:256
    \item LSTM层数:3
    \item Dropout比率:0.3
    \item 全连接层隐藏维度:128
    \item 批次大小:32
    \item 学习率:0.0003
    \item 优化器:Adam(带L2正则化,权重衰减=1e-5)
    \item 损失函数:均方误差(MSE)
\end{itemize}

\subsection{训练策略}

模型训练采用以下策略:
\begin{itemize}
    \item 使用早停机制(Early Stopping),耐心值为20,当验证损失连续20个epoch不下降时停止训练
    \item 使用学习率衰减策略(ReduceLROnPlateau),当验证损失不下降时降低学习率
    \item 保存验证损失最低的模型作为最佳模型
    \item 最大训练轮数为300轮
\end{itemize}

模型最终训练了66个epoch,训练过程如图\ref{fig:training_history}所示。

\section{模型评估}

\subsection{评估指标}

为了全面评估模型性能,我们选择了多个评估指标:

\subsubsection{回归指标}

\begin{itemize}
    \item \textbf{均方误差(MSE)}:
    \begin{equation}
    \text{MSE} = \frac{1}{n}\sum_{i=1}^{n}(y_i - \hat{y}_i)^2
    \end{equation}
    
    \item \textbf{平均绝对误差(MAE)}:
    \begin{equation}
    \text{MAE} = \frac{1}{n}\sum_{i=1}^{n}|y_i - \hat{y}_i|
    \end{equation}
    
    \item \textbf{均方根误差(RMSE)}:
    \begin{equation}
    \text{RMSE} = \sqrt{\frac{1}{n}\sum_{i=1}^{n}(y_i - \hat{y}_i)^2}
    \end{equation}
    
    \item \textbf{决定系数(R²)}:
    \begin{equation}
    R^2 = 1 - \frac{\sum_{i=1}^{n}(y_i - \hat{y}_i)^2}{\sum_{i=1}^{n}(y_i - \bar{y})^2}
    \end{equation}
    其中$\bar{y}$为目标值的均值。R²越接近1,说明模型解释能力越强。
    
    \item \textbf{平均绝对百分比误差(MAPE)}:
    \begin{equation}
    \text{MAPE} = \frac{100}{n}\sum_{i=1}^{n}\left|\frac{y_i - \hat{y}_i}{y_i}\right|
    \end{equation}
\end{itemize}

\subsubsection{准确率指标}

为了更直观地评估模型性能,我们还计算了基于相对误差阈值的准确率:
\begin{itemize}
    \item 准确率(5\%):相对误差小于5\%的样本比例
    \item 准确率(10\%):相对误差小于10\%的样本比例
    \item 准确率(20\%):相对误差小于20\%的样本比例
\end{itemize}

\subsection{测试集评估结果}

模型在测试集上的评估结果如表\ref{tab:results}所示。

\begin{table}[H]
\centering
\caption{模型在测试集上的评估结果}
\label{tab:results}
\begin{tabular}{@{}lc@{}}
\toprule
\textbf{评估指标} & \textbf{数值} \\
\midrule
MSE & 252.77 \\
MAE & 10.13 \\
RMSE & 15.90 \\
R² & 0.505 \\
MAPE & 2.28\% \\
\midrule
准确率(5\%) & 88.94\% \\
准确率(10\%) & 97.61\% \\
准确率(20\%) & 100.00\% \\
\bottomrule
\end{tabular}
\end{table}

从评估结果可以看出:
\begin{itemize}
    \item R²达到0.505,说明模型能够解释约50.5\%的数据变异,具有较好的预测能力
    \item MAE为10.13,RMSE为15.90,说明平均预测误差较小
    \item MAPE为2.28\%,说明相对误差很小
    \item 97.61\%的样本预测误差在10\%以内,88.94\%的样本预测误差在5\%以内,准确率较高
\end{itemize}

\subsection{可视化分析}

\subsubsection{预测结果对比}

图\ref{fig:predictions}展示了模型在测试集上的预测结果与真实值的对比。从图中可以看出,预测值与真实值基本吻合,模型能够较好地捕捉温度变化的趋势。

\begin{figure}[H]
\centering
\includegraphics[width=0.9\textwidth]{visualizations/predictions.png}
\caption{测试集预测结果对比}
\label{fig:predictions}
\end{figure}

\subsubsection{预测散点图}

图\ref{fig:scatter}展示了预测值与真实值的散点图。理想情况下,所有点应该落在$y=x$直线上。从图中可以看出,大部分点都集中在对角线附近,说明模型预测较为准确。

\begin{figure}[H]
\centering
\includegraphics[width=0.8\textwidth]{visualizations/prediction_scatter.png}
\caption{预测值vs真实值散点图}
\label{fig:scatter}
\end{figure}

\subsubsection{误差分布}

图\ref{fig:error_dist}展示了预测误差的分布情况。误差分布接近正态分布,均值接近0,说明模型预测无系统性偏差。

\begin{figure}[H]
\centering
\includegraphics[width=0.8\textwidth]{visualizations/error_distribution.png}
\caption{预测误差分布}
\label{fig:error_dist}
\end{figure}

\subsubsection{残差分析}

图\ref{fig:residuals}展示了残差图,用于检查模型的假设是否成立。从图中可以看出,残差基本随机分布,无明显模式,说明模型假设合理。

\begin{figure}[H]
\centering
\includegraphics[width=0.8\textwidth]{visualizations/residuals.png}
\caption{残差分析图}
\label{fig:residuals}
\end{figure}

\section{结论}

本实验成功构建了一个基于LSTM和注意力机制的时间序列预测模型,用于预测室外温度变化。通过以下关键步骤:

\begin{enumerate}
    \item \textbf{数据预处理}:对20个气象特征进行标准化处理,使用滑动窗口方法将时间序列转换为监督学习问题,按时间顺序划分训练集和测试集
    \item \textbf{模型设计}:设计了包含3层LSTM、注意力机制和3层全连接网络的深度学习模型,能够有效捕捉时间序列的长期依赖关系
    \item \textbf{模型评估}:在测试集上使用多个评估指标进行评估,R²达到0.505,97.61\%的样本预测误差在10\%以内,模型表现良好
\end{enumerate}

实验结果表明,该模型能够较好地预测室外温度变化,具有一定的实用价值。未来可以进一步优化模型结构,增加特征工程,或尝试集成学习方法以提升模型性能。

\end{document}

