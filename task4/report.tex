\documentclass[12pt,a4paper]{article}
\usepackage[UTF8]{ctex}
\usepackage{amsmath}
\usepackage{amssymb}
\usepackage{graphicx}
\usepackage{booktabs}
\usepackage{multirow}
\usepackage{hyperref}
\usepackage{geometry}
\usepackage{float}
\usepackage{algorithm}
\usepackage{algorithmic}
\usepackage{subcaption}
\usepackage{xcolor}

\geometry{left=2.5cm,right=2.5cm,top=2.5cm,bottom=2.5cm}

\title{\textbf{无监督疾病判断任务报告\\基于异常检测的甲状腺疾病识别}}
\author{}
\date{}

\begin{document}

\maketitle

\section{任务概述}

本任务旨在利用无监督学习方法对甲状腺疾病进行判断。数据集包含3772个样本,每个样本有6个特征维度。训练集包含1839个正常样本,测试集包含1933个样本(其中94个患病样本,1839个正常样本)。由于训练集仅包含正常样本,这是一个典型的\textbf{异常检测(Anomaly Detection)}问题。

\section{问题分析}

\subsection{数据特点}

通过数据探索,我们发现数据集具有以下特点:

\begin{enumerate}
    \item \textbf{单类训练数据}:训练集仅包含正常样本(标签为0),这是异常检测的典型场景
    \item \textbf{类别不平衡}:测试集中患病样本占比约4.86\%,正常样本占95.14\%
    \item \textbf{数据已标准化}:所有特征的均值接近0,标准差接近1
    \item \textbf{无缺失值}:数据质量良好,不需要额外的数据清洗
    \item \textbf{中等维度}:6个特征维度,适合大多数异常检测算法
\end{enumerate}

\subsection{问题建模}

本问题可以形式化为:

\begin{itemize}
    \item \textbf{训练阶段}:给定正常样本集 $\mathcal{X}_{\text{train}} = \{x_i\}_{i=1}^{1839}$,学习正常样本的分布 $P(x|\text{正常})$
    \item \textbf{测试阶段}:对于新样本 $x \in \mathcal{X}_{\text{test}}$,计算异常分数 $s(x)$,如果 $s(x) > \theta$(阈值),则判定为患病
\end{itemize}

关键挑战在于:\textbf{如何在只有正常样本的情况下,准确识别出偏离正常模式的异常样本(患病样本)}。

\section{算法选择与设计思路}

\subsection{算法选择理由}

基于数据特点和问题性质,我选择了5种经典的无监督异常检测算法:

\subsubsection{孤立森林(Isolation Forest)}

\textbf{原理}:异常样本更容易被"孤立"。通过随机选择特征和切分点构建决策树,异常点需要更少的切分次数就能被孤立。

\textbf{优点}:
\begin{itemize}
    \item 不需要假设数据分布
    \item 对高维数据效果好
    \item 训练速度快,时间复杂度为 $O(n\log n)$
    \item 对异常值敏感度高
\end{itemize}

\textbf{适用性}:非常适合本任务,因为患病样本在特征空间中与正常样本分布不同,容易被孤立。

\subsubsection{单类支持向量机(One-Class SVM)}

\textbf{原理}:在高维空间中找到一个超平面,将正常样本包围起来。使用核技巧将数据映射到高维空间,在边界外的样本被认为是异常。

\textbf{优点}:
\begin{itemize}
    \item 理论基础扎实(基于统计学习理论)
    \item 适合高维数据
    \item 对噪声鲁棒
    \item 可以通过核函数处理非线性问题
\end{itemize}

\textbf{适用性}:适合医疗数据,因为正常样本可能在特征空间中形成一个紧密的区域。

\subsubsection{局部离群因子(Local Outlier Factor, LOF)}

\textbf{原理}:基于局部密度的异常检测。比较每个样本与其邻居的局部密度,密度明显低于邻居的样本被认为是异常。

\textbf{优点}:
\begin{itemize}
    \item 能够发现局部异常(在全局看正常但在局部看异常的样本)
    \item 不需要假设数据分布
    \item 对不同密度的聚类效果好
\end{itemize}

\textbf{适用性}:适合发现那些在某些特征组合下异常的患病样本。

\subsubsection{椭圆包络(Elliptic Envelope)}

\textbf{原理}:假设正常数据服从多元高斯分布,通过鲁棒协方差估计拟合一个椭圆包络。在椭圆外的样本被认为是异常。

\textbf{优点}:
\begin{itemize}
    \item 对多元正态分布数据效果好
    \item 对异常值鲁棒(使用鲁棒协方差估计)
    \item 计算效率高
    \item 提供统计学解释
\end{itemize}

\textbf{适用性}:适合医疗数据,因为正常生理指标通常服从正态分布。

\subsubsection{高斯混合模型(Gaussian Mixture Model, GMM)}

\textbf{原理}:假设正常数据由多个高斯分布混合而成,计算每个样本的对数似然概率,概率低的样本被认为是异常。

\textbf{优点}:
\begin{itemize}
    \item 能够建模复杂的多模态分布
    \item 提供概率解释
    \item 适合聚类结构明显的数据
    \item 可以自动发现数据中的子群体
\end{itemize}

\textbf{适用性}:适合可能存在多个正常亚型的医疗数据。

\subsection{算法设计细节}

\subsubsection{污染率(Contamination)设置}

污染率是异常检测算法的重要超参数,表示数据中异常样本的预期比例。我使用测试集的真实患病比例(4.86\%)作为污染率,这是一个合理的先验估计。

\subsubsection{异常分数计算}

为了统一评估,所有算法都实现了 \texttt{get\_anomaly\_score()} 方法:

\begin{itemize}
    \item \textbf{Isolation Forest, One-Class SVM, LOF, Elliptic Envelope}:使用 \texttt{-decision\_function()},使得分数越高越异常
    \item \textbf{GMM}:使用负对数似然 $-\log P(x)$,概率低的样本异常分数高
\end{itemize}

\subsubsection{模型训练流程}

\begin{algorithm}[H]
\caption{异常检测模型训练与评估}
\begin{algorithmic}[1]
\STATE \textbf{输入}: 训练集 $\mathcal{X}_{\text{train}}$(只包含正常样本),测试集 $\mathcal{X}_{\text{test}}$,真实标签 $\mathcal{Y}_{\text{test}}$
\STATE \textbf{输出}: 预测标签 $\hat{\mathcal{Y}}_{\text{test}}$,异常分数 $\mathcal{S}_{\text{test}}$
\STATE
\STATE // 训练阶段
\STATE 在 $\mathcal{X}_{\text{train}}$ 上训练模型,学习正常样本的分布
\STATE
\STATE // 测试阶段
\FOR{每个测试样本 $x \in \mathcal{X}_{\text{test}}$}
    \STATE 计算异常分数 $s(x)$
    \IF{$s(x) > \theta$}
        \STATE 预测为患病($\hat{y} = 1$)
    \ELSE
        \STATE 预测为正常($\hat{y} = 0$)
    \ENDIF
\ENDFOR
\STATE
\STATE // 评估阶段
\STATE 计算准确率、精确率、召回率、F1分数、ROC-AUC等指标
\end{algorithmic}
\end{algorithm}

\section{评估指标选择}

考虑到这是一个类别不平衡的异常检测问题,我选择了以下评估指标:

\subsection{混淆矩阵}

\begin{table}[H]
\centering
\begin{tabular}{cc|cc}
\hline
& & \multicolumn{2}{c}{\textbf{预测}} \\
& & 正常 & 患病 \\
\hline
\multirow{2}{*}{\textbf{实际}} & 正常 & TN & FP \\
& 患病 & FN & TP \\
\hline
\end{tabular}
\end{table}

\subsection{关键指标}

\begin{itemize}
    \item \textbf{准确率(Accuracy)}: $\frac{TP + TN}{TP + TN + FP + FN}$
    \begin{itemize}
        \item 衡量整体预测正确的比例
    \end{itemize}

    \item \textbf{精确率(Precision)}: $\frac{TP}{TP + FP}$
    \begin{itemize}
        \item 衡量预测为患病的样本中真正患病的比例
        \item 在医疗场景中,高精确率意味着减少误诊(将健康人误诊为患病)
    \end{itemize}

    \item \textbf{召回率(Recall)}: $\frac{TP}{TP + FN}$
    \begin{itemize}
        \item 衡量所有患病样本中被正确识别的比例
        \item \textcolor{red}{\textbf{这是医疗诊断中最重要的指标}},因为漏诊(FN)的后果非常严重
    \end{itemize}

    \item \textbf{F1分数}: $2 \cdot \frac{\text{Precision} \cdot \text{Recall}}{\text{Precision} + \text{Recall}}$
    \begin{itemize}
        \item 精确率和召回率的调和平均,平衡两者的权重
    \end{itemize}

    \item \textbf{ROC-AUC(受试者工作特征曲线下面积)}
    \begin{itemize}
        \item 衡量模型在不同阈值下的分类能力
        \item 不受类别不平衡影响,是评估异常检测模型的重要指标
    \end{itemize}
\end{itemize}

\subsection{指标选择理由}

在医疗诊断场景中,\textbf{召回率}是最关键的指标,因为:
\begin{enumerate}
    \item 漏诊(将患病样本判断为正常)的后果远比误诊严重
    \item 即使精确率较低(一些正常样本被误判为患病),也可以通过后续检查进行排除
    \item 高召回率确保尽可能多的患病样本被发现并得到治疗
\end{enumerate}

因此,我们的目标是在保持较高召回率的前提下,尽可能提高精确率,即优化F1分数和ROC-AUC。

\section{实验结果}

\subsection{整体性能对比}

表~\ref{tab:results}展示了五种算法在测试集上的性能:

\begin{table}[H]
\centering
\caption{各算法性能对比}
\label{tab:results}
\begin{tabular}{lccccc}
\toprule
\textbf{模型} & \textbf{准确率} & \textbf{精确率} & \textbf{召回率} & \textbf{F1分数} & \textbf{ROC-AUC} \\
\midrule
Isolation Forest & \textbf{0.9395} & 0.4433 & \textbf{0.9574} & \textbf{0.6061} & \textbf{0.9787} \\
One-Class SVM & 0.9317 & 0.4087 & 0.9043 & 0.5629 & 0.9606 \\
LOF & \textbf{0.9421} & \textbf{0.4505} & 0.8723 & 0.5942 & 0.9658 \\
Elliptic Envelope & 0.9374 & 0.4335 & 0.9362 & 0.5926 & 0.9746 \\
GMM & 0.9353 & 0.4244 & 0.9255 & 0.5819 & 0.9718 \\
\bottomrule
\end{tabular}
\end{table}

\subsection{最佳模型:Isolation Forest}

从实验结果可以看出,\textbf{Isolation Forest(孤立森林)}在所有关键指标上表现最优:

\begin{itemize}
    \item \textbf{F1分数最高}:0.6061,说明在精确率和召回率之间达到了最好的平衡
    \item \textbf{ROC-AUC最高}:0.9787,表明模型具有优秀的分类能力
    \item \textbf{召回率最高}:0.9574,意味着94个患病样本中成功识别出90个,只有4个漏诊
    \item 准确率:0.9395,整体预测准确
\end{itemize}

\subsection{混淆矩阵分析}

Isolation Forest的混淆矩阵:

\begin{table}[H]
\centering
\begin{tabular}{cc|cc}
\hline
& & \multicolumn{2}{c}{\textbf{预测}} \\
& & 正常 & 患病 \\
\hline
\multirow{2}{*}{\textbf{实际}} & 正常 & 1726 & 113 \\
& 患病 & 4 & 90 \\
\hline
\end{tabular}
\end{table}

\textbf{关键观察}:
\begin{itemize}
    \item \textcolor{red}{\textbf{True Positive (TP) = 90}}:成功识别90个患病样本
    \item \textcolor{red}{\textbf{False Negative (FN) = 4}}:仅漏诊4个患病样本(4.26\%)
    \item False Positive (FP) = 113:将113个正常样本误判为患病(6.14\%)
    \item True Negative (TN) = 1726:正确识别1726个正常样本
\end{itemize}

从医疗角度看,这是一个非常好的结果:
\begin{enumerate}
    \item 高召回率(95.74\%)确保绝大多数患者被发现
    \item 虽然精确率相对较低(44.33\%),但误诊的患者可以通过进一步检查排除
    \item 相比漏诊,这种误诊是可以接受的代价
\end{enumerate}

\subsection{其他模型分析}

\subsubsection{LOF(局部离群因子)}

\begin{itemize}
    \item 准确率最高(0.9421),精确率最高(0.4505)
    \item 召回率相对较低(0.8723),意味着12个患病样本被漏诊
    \item 适合对精确率要求较高的场景
\end{itemize}

\subsubsection{One-Class SVM}

\begin{itemize}
    \item 性能相对较弱,召回率为0.9043,9个患病样本被漏诊
    \item 可能原因:RBF核的超参数需要进一步调优
    \item 训练时间较长,不适合大规模数据
\end{itemize}

\subsubsection{Elliptic Envelope 和 GMM}

\begin{itemize}
    \item 性能介于Isolation Forest和One-Class SVM之间
    \item 适合假设数据服从高斯分布的场景
    \item GMM可以捕捉多模态分布,但在本任务中优势不明显
\end{itemize}

\section{算法选择总结}

\subsection{为什么选择Isolation Forest?}

基于实验结果,我推荐使用\textbf{Isolation Forest}作为甲状腺疾病判断的最佳算法,理由如下:

\begin{enumerate}
    \item \textbf{最高的召回率(0.9574)}
    \begin{itemize}
        \item 在医疗诊断中,漏诊的代价远大于误诊
        \item 高召回率确保尽可能多的患者被发现并得到治疗
    \end{itemize}

    \item \textbf{最佳的F1分数(0.6061)和ROC-AUC(0.9787)}
    \begin{itemize}
        \item 在精确率和召回率之间达到最佳平衡
        \item 优秀的分类能力,能够有效区分患病和正常样本
    \end{itemize}

    \item \textbf{算法优势}
    \begin{itemize}
        \item 不需要假设数据分布,适应性强
        \item 对异常值高度敏感,适合异常检测任务
        \item 训练速度快,时间复杂度 $O(n\log n)$
        \item 可解释性强:通过平均路径长度判断异常
    \end{itemize}

    \item \textbf{实用性}
    \begin{itemize}
        \item 超参数少,容易调优
        \item 对特征尺度不敏感
        \item 适合在线学习和增量更新
    \end{itemize}
\end{enumerate}

\subsection{Isolation Forest工作原理}

Isolation Forest通过以下步骤检测异常:

\begin{enumerate}
    \item \textbf{构建孤立树}:随机选择特征和切分点,递归分割数据
    \item \textbf{计算路径长度}:对于每个样本,记录从根节点到叶节点的路径长度
    \item \textbf{异常分数}:路径长度越短,越容易被孤立,异常分数越高
    \item \textbf{集成学习}:构建多棵孤立树,对路径长度取平均,提高鲁棒性
\end{enumerate}

\textbf{核心思想}:异常样本在特征空间中与正常样本距离较远,因此更容易被孤立(需要更少的分割次数)。

\section{结论与展望}

\subsection{主要结论}

\begin{enumerate}
    \item 本任务是一个典型的异常检测问题,训练集仅包含正常样本,需要识别测试集中的患病样本
    \item 实验对比了5种经典的无监督异常检测算法,\textbf{Isolation Forest}表现最佳
    \item Isolation Forest在召回率(0.9574)、F1分数(0.6061)和ROC-AUC(0.9787)三个关键指标上均优于其他算法
    \item 所有算法都实现了较高的准确率(>93\%),但在精确率和召回率的平衡上有所差异
\end{enumerate}

\subsection{医疗应用建议}

在实际医疗应用中,建议采用以下策略:

\begin{enumerate}
    \item \textbf{两阶段筛查}:
    \begin{itemize}
        \item 第一阶段:使用Isolation Forest进行初步筛查,确保高召回率
        \item 第二阶段:对预测为患病的样本进行进一步的临床检查,排除误诊
    \end{itemize}

    \item \textbf{阈值调整}:
    \begin{itemize}
        \item 根据实际医疗需求调整异常分数阈值
        \item 如果需要更高的召回率(减少漏诊),可以降低阈值
        \item 如果需要更高的精确率(减少误诊),可以提高阈值
    \end{itemize}

    \item \textbf{集成多模型}:
    \begin{itemize}
        \item 结合多个算法的预测结果,采用投票或加权平均的方式
        \item 例如:只有当Isolation Forest和LOF都预测为患病时,才判定为患病(提高精确率)
    \end{itemize}
\end{enumerate}

\subsection{未来改进方向}

\begin{enumerate}
    \item \textbf{半监督学习}:如果能够获取少量患病样本,可以采用半监督异常检测算法
    \item \textbf{深度学习方法}:尝试Autoencoder、Variational Autoencoder等深度学习方法
    \item \textbf{特征工程}:分析特征重要性,进行特征选择和特征交互
    \item \textbf{超参数优化}:使用网格搜索或贝叶斯优化进一步调优模型参数
    \item \textbf{集成学习}:结合多个异常检测算法的优势,构建集成模型
\end{enumerate}

\section*{附录:实验环境}

\begin{itemize}
    \item \textbf{编程语言}:Python 3.13
    \item \textbf{主要库}:
    \begin{itemize}
        \item scikit-learn 1.7.2(机器学习算法)
        \item pandas 2.3.3(数据处理)
        \item numpy 2.3.5(数值计算)
        \item matplotlib 3.10.7(数据可视化)
    \end{itemize}
    \item \textbf{数据集}:Thyroid甲状腺疾病数据集
    \begin{itemize}
        \item 训练集:1839个正常样本
        \item 测试集:1933个样本(94个患病,1839个正常)
        \item 特征维度:6
    \end{itemize}
\end{itemize}

\end{document}
