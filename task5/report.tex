\documentclass[12pt,a4paper]{article}
\usepackage[UTF8]{ctex}
\usepackage{amsmath}
\usepackage{amsfonts}
\usepackage{amssymb}
\usepackage{graphicx}
\usepackage{float}
\usepackage{hyperref}
\usepackage{listings}
\usepackage{xcolor}
\usepackage{booktabs}
\usepackage{geometry}
\geometry{left=2.5cm,right=2.5cm,top=2.5cm,bottom=2.5cm}

% 代码样式设置
\lstset{
    language=Python,
    basicstyle=\ttfamily\small,
    keywordstyle=\color{blue},
    commentstyle=\color{green!60!black},
    stringstyle=\color{red},
    numbers=left,
    numberstyle=\tiny\color{gray},
    stepnumber=1,
    numbersep=5pt,
    backgroundcolor=\color{gray!10},
    showspaces=false,
    showstringspaces=false,
    showtabs=false,
    frame=single,
    rulecolor=\color{black},
    tabsize=2,
    captionpos=b,
    breaklines=true,
    breakatwhitespace=false,
    escapeinside={\%*}{*)}
}

\title{跨模态异常检测统一框架分析}
\author{数据挖掘导论2025课程作业}
\date{\today}

\begin{document}

\maketitle

\tableofcontents
\newpage

\section{任务概述}

本报告针对数据挖掘导论课程中的Task5进行分析。Task5要求我们分析Task2(图像异常检测)和Task4(无监督疾病判断)这两种不同模态的异常检测任务的异同,并提出可能的统一框架。通过对两种任务的深入分析,我们将探讨是否可以设计一种通用的方法来同时处理图像和表格数据的异常检测问题。

\section{Task2与Task4任务回顾}

\subsection{Task2:图像异常检测}

Task2是一个图像异常检测任务,使用包含榛子(hazelnut)和拉链(zipper)两个类别的数据集。每个类别包含200张正常样本和50张异常样本。该任务主要使用自编码器(Autoencoder)和变分自编码器(VAE)作为异常检测模型,通过计算重构误差来识别异常样本。

\subsection{Task4:无监督疾病判断}

Task4是一个无监督疾病判断任务,使用甲状腺疾病数据集。数据集包含3772个样本,每个样本有6个特征维度。训练集仅包含正常样本(标签为0),测试集包含正常样本和患病样本(标签为1)。该任务使用了多种无监督异常检测算法,包括孤立森林(Isolation Forest)、单类支持向量机(One-Class SVM)、局部离群因子(LOF)等,最终确定孤立森林为最佳算法。

\section{两种模态异常检测任务的异同}

\subsection{相同点}

\begin{enumerate}
    \item \textbf{问题本质相同}:都是异常检测问题,目标是识别与正常样本分布显著不同的异常样本
    \item \textbf{训练数据特点}:都主要使用正常样本来训练模型
    \item \textbf{评估指标相似}:都使用准确率、精确率、召回率、F1分数、ROC-AUC等指标进行评估
    \item \textbf{阈值依赖}:都需要设定合适的阈值来区分正常和异常样本
    \item \textbf{任务类型}:都属于单类分类问题(One-class classification)
\end{enumerate}

\subsection{不同点}

\begin{enumerate}
    \item \textbf{数据类型差异}:
    \begin{itemize}
        \item Task2:图像数据(二维结构数据),高维、非结构化
        \item Task4:表格数据(6个特征维度),低维、结构化
    \end{itemize}

    \item \textbf{特征提取方式}:
    \begin{itemize}
        \item Task2:使用深度学习模型(自编码器)自动提取特征
        \item Task4:直接使用原始特征,或进行简单的特征选择
    \end{itemize}

    \item \textbf{采用的算法}:
    \begin{itemize}
        \item Task2:主要使用深度学习方法(自编码器、变分自编码器)
        \item Task4:主要使用传统机器学习方法(Isolation Forest、One-Class SVM、LOF等)
    \end{itemize}

    \item \textbf{异常定义}:
    \begin{itemize}
        \item Task2:图像中的局部异常(如缺陷、损坏)
        \item Task4:生理指标的整体异常模式(患病状态)
    \end{itemize}

    \item \textbf{数据规模}:
    \begin{itemize}
        \item Task2:每个类别几百张图像
        \item Task4:几千个样本
    \end{itemize}

    \item \textbf{异常评分机制}:
    \begin{itemize}
        \item Task2:主要基于重构误差
        \item Task4:基于各种统计或几何距离度量
    \end{itemize}
\end{enumerate}

\section{统一框架设计}

\subsection{1. 多模态特征提取层}

\begin{figure}[H]
    \centering
    \begin{tabular}{c c c c c}
        输入数据 & \multicolumn{1}{c}{---$\boldsymbol{\rightarrow}$---$\boldsymbol{\uparrow}$---$\boldsymbol{\rightarrow}$} & 图像特征提取器(CNN backbone) & \multicolumn{1}{c}{---$\boldsymbol{\rightarrow}$} & 特征融合 \\
        & \multicolumn{1}{c}{$\boldsymbol{\vert}$} & & \multicolumn{1}{c}{$\boldsymbol{\vert}$} & \\
        & \multicolumn{1}{c}{---$\boldsymbol{\rightarrow}$} & 表格特征提取器(MLP/特征选择) & \multicolumn{1}{c}{---$\boldsymbol{\rightarrow}$} & 
    \end{tabular}
    \caption{多模态特征提取层结构}
    \label{fig:feature_extraction}
\end{figure}

\textbf{设计思路}:
\begin{itemize}
    \item 为不同模态设计专门的特征提取器
    \item 图像模态:使用预训练的CNN(如ResNet、VGG的编码器部分)提取深层特征
    \item 表格模态:使用MLP或特征选择方法提取关键特征
    \item 确保两种模态的特征维度一致,便于后续处理
\end{itemize}

\subsection{2. 统一异常检测核心层}

\begin{figure}[H]
    \centering
    \begin{tabular}{c c c c c c c}
        特征融合 & \multicolumn{1}{c}{$\boldsymbol{\rightarrow}$} & 共享表示空间 & \multicolumn{1}{c}{$\boldsymbol{\rightarrow}$} & 异常检测器 & \multicolumn{1}{c}{$\boldsymbol{\rightarrow}$} & 异常分数 
    \end{tabular}
    \caption{统一异常检测核心层结构}
    \label{fig:core_detection}
\end{figure}

\textbf{核心算法选择}:
\begin{itemize}
    \item \textbf{变分自编码器(VAE)}:能够有效处理高维和低维数据,引入概率建模
    \item \textbf{隔离森林(Isolation Forest)}:可解释性强,适合各种数据类型
    \item \textbf{自注意力机制}:可以捕获不同特征之间的关系
\end{itemize}

\subsection{3. 自适应阈值层}

\begin{figure}[H]
    \centering
    \begin{tabular}{c c c c c}
        异常分数 & \multicolumn{1}{c}{$\boldsymbol{\rightarrow}$} & 自适应阈值确定 & \multicolumn{1}{c}{$\boldsymbol{\rightarrow}$} & 二分类结果 \\
    \end{tabular}
    \caption{自适应阈值层结构}
    \label{fig:threshold_layer}
\end{figure}

\textbf{设计思路}:
\begin{itemize}
    \item 基于验证集动态调整阈值
    \item 考虑类别不平衡问题
    \item 可根据具体应用场景的需求(如医疗场景更关注召回率)调整阈值
\end{itemize}

\section{具体实现方案}

\subsection{方案一:基于多模态VAE的统一框架}

\begin{lstlisting}[caption=多模态VAE实现]
class MultiModalVAE:
    def __init__(self, image_dims, tabular_dims, latent_dim=64):
        # 图像编码器
        self.image_encoder = ImageEncoder(image_dims, latent_dim)
        # 表格编码器
        self.tabular_encoder = TabularEncoder(tabular_dims, latent_dim)
        # 共享解码器
        self.decoder = SharedDecoder(latent_dim, image_dims, tabular_dims)
    
    def forward(self, x_image=None, x_tabular=None):
        # 根据输入模态选择编码器
        if x_image is not None:
            z_mean, z_logvar = self.image_encoder(x_image)
        elif x_tabular is not None:
            z_mean, z_logvar = self.tabular_encoder(x_tabular)
        
        # 重参数化
        z = self.reparameterize(z_mean, z_logvar)
        
        # 解码
        recon_image, recon_tabular = self.decoder(z)
        
        # 计算损失
        loss = self.compute_loss(x_image, x_tabular, recon_image, recon_tabular, z_mean, z_logvar)
        
        return loss, recon_image, recon_tabular, z_mean, z_logvar
    
    def compute_anomaly_score(self, x, type='image'):
        # 计算异常分数(重构误差 + KL散度)
        if type == 'image':
            _, recon, _, z_mean, z_logvar = self.forward(x_image=x)
            mse_loss = F.mse_loss(recon, x)
        else:
            _, _, recon, z_mean, z_logvar = self.forward(x_tabular=x)
            mse_loss = F.mse_loss(recon, x)
        
        kl_loss = -0.5 * torch.sum(1 + z_logvar - z_mean.pow(2) - z_logvar.exp())
        
        return mse_loss + kl_loss
\end{lstlisting}

\subsection{方案二:基于集成学习的统一框架}

\begin{lstlisting}[caption=集成学习异常检测器实现]
class EnsembleAnomalyDetector:
    def __init__(self, models=None):
        # 支持多种异常检测算法
        if models is None:
            self.models = {
                'iforest': IsolationForest(),
                'ocsvm': OneClassSVM(),
                'vae': VariationalAutoencoder()
            }
        else:
            self.models = models
    
    def fit(self, X, type='tabular'):
        # 根据数据类型选择合适的特征预处理
        if type == 'image':
            X_features = self.extract_image_features(X)
        else:
            X_features = X  # 表格数据直接使用
        
        # 训练各个模型
        for name, model in self.models.items():
            model.fit(X_features)
    
    def predict(self, X, type='tabular', threshold=None):
        # 提取特征
        if type == 'image':
            X_features = self.extract_image_features(X)
        else:
            X_features = X
        
        # 集成各个模型的异常分数
        scores = []
        for name, model in self.models.items():
            if hasattr(model, 'decision_function'):
                score = -model.decision_function(X_features)  # 使分数越高越异常
            elif hasattr(model, 'score_samples'):
                score = -model.score_samples(X_features)
            elif hasattr(model, 'compute_anomaly_score'):
                score = model.compute_anomaly_score(X_features)
            scores.append(score)
        
        # 计算平均异常分数
        avg_score = np.mean(scores, axis=0)
        
        # 根据阈值预测
        if threshold is not None:
            return (avg_score > threshold).astype(int)
        
        return avg_score
\end{lstlisting}

\section{可行性分析}

\subsection{为什么可以统一?}

\begin{enumerate}
    \item \textbf{异常检测的本质相同}:都是识别偏离正常分布的样本
    \item \textbf{特征表示可以统一}:通过合适的特征提取器,可以将不同模态的数据映射到统一的特征空间
    \item \textbf{异常评分机制可以通用}:重构误差、密度估计、隔离路径长度等异常评分机制可以应用于不同模态
    \item \textbf{深度学习技术的进步}:现代深度学习模型能够处理多种数据类型
\end{enumerate}

\subsection{可能的挑战}

\begin{enumerate}
    \item \textbf{模态差异}:图像和表格数据的特性差异较大,需要专门的特征提取策略
    \item \textbf{计算复杂度}:处理多种模态可能增加计算负担
    \item \textbf{超参数调优}:需要为不同模态调整合适的超参数
    \item \textbf{数据不平衡}:不同模态的异常样本比例可能不同
\end{enumerate}

\section{实施建议}

\subsection{分阶段实施}

\begin{enumerate}
    \item \textbf{第一阶段}:分别为两种模态实现各自的异常检测模型
    \item \textbf{第二阶段}:设计统一的特征提取和融合机制
    \item \textbf{第三阶段}:实现端到端的统一框架
\end{enumerate}

\subsection{模型选择}

\begin{enumerate}
    \item \textbf{对于图像模态}:保留自编码器/VAE架构
    \item \textbf{对于表格模态}:可以考虑将表格数据转换为二维表示后使用CNN,或使用MLP
    \item \textbf{核心检测算法}:建议使用VAE或集成学习方法
\end{enumerate}

\subsection{评估策略}

\begin{enumerate}
    \item 使用统一的评估指标(AUC-ROC、F1分数等)
    \item 分别评估在两种模态上的性能
    \item 分析模型在不同异常类型上的表现
\end{enumerate}

\section{结论}

基于上述分析,我们认为可以设计一个统一的框架来处理图像和表格数据的异常检测任务。通过多模态特征提取、统一的表示空间和灵活的异常评分机制,可以有效处理两种不同模态的数据。这种统一框架不仅能够简化系统设计,还可以通过知识迁移提高模型的性能和泛化能力。

当然,在实际实施过程中,需要针对具体的数据特点和应用场景进行适当的调整和优化,以达到最佳的检测效果。

\section{参考文献}

\begin{enumerate}
    \item Vincent, P., et al. (2008). Extracting and composing robust features with denoising autoencoders. \textit{In} \textit{Proceedings of the 25th International Conference on Machine Learning (ICML)}, 1096--1103.
    \item Kingma, D. P., \& Welling, M. (2013). Auto-encoding variational bayes. \textit{arXiv preprint arXiv:1312.6114 [stat.ML]}.
    \item Goodfellow, I., et al. (2014). Generative adversarial nets. \textit{In} \textit{Advances in Neural Information Processing Systems (NeurIPS)}, 27, 2672--2680.
    \item Schölkopf, B., et al. (2001). Estimating the support of a high-dimensional distribution. \textit{Neural Computation}, 13(7), 1443--1471.
    \item Ruff, L., et al. (2021). Deep one-class classification. \textit{Journal of Machine Learning Research}, 22(43), 1--65.
    \item Chalapathy, R., \& Chawla, S. (2019). Deep learning for anomaly detection: A survey. \textit{arXiv preprint arXiv:1901.03407 [cs.LG]}.
    \item Liu, F. T., Ting, K. M., \& Zhou, Z. H. (2008). Isolation forest. \textit{In} \textit{2008 Eighth IEEE International Conference on Data Mining (ICDM)}, 413--422.
    \item Breunig, M. M., Kriegel, H. P., Ng, R. T., \& Sander, J. (2000). LOF: identifying density-based local outliers. \textit{ACM SIGMOD Record}, 29(2), 93--104.
\end{enumerate}

\end{document}